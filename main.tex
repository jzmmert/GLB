\documentclass{article}
\usepackage[utf8]{inputenc}
\usepackage{amsmath}
\usepackage{amssymb}
\usepackage{color}
\usepackage{amsthm}

 
\newtheorem{theorem}{Theorem}
\newtheorem{corollary}{Corollary}[theorem]
\newtheorem{lemma}[theorem]{Lemma}
\newtheorem{proposition}[theorem]{Proposition}

 
\theoremstyle{definition}
\newtheorem{definition}{Definition}[section]
\newcommand{\EV}[1] {
  \mathbb{E}\left[#1\right]}
\newcommand{\sign} {
  \operatorname{sign}}
  \DeclareMathOperator\caret{\raisebox{1ex}{$\scriptstyle\wedge$}}

\title{Generalized Linear Bandits}
\author{julian.zimmert }
\date{July 2016}

\begin{document}

\maketitle
\section{Generalized Linear Model}
The arms are denoted as $x_i \in \mathbb{R}^d$, whose features are specific to each arm and known to the agent.\\
There exists a known link function $\mu:\mathbb{R}\rightarrow\mathbb{R}$. $\mu$ is monotonously increasing and satisfies for any $x<y$: $c_\mu (y-x)\leq\mu(y)-\mu(x)\leq \kappa_\mu(y-x)$. For some unknown parameter vector $\theta_*\in\mathbb{R}^d$ the payoffs received at time $t$ are
\begin{align}
R_t = \mu(x_t^T\theta_*) + \eta_t,
\end{align}
where $\eta_t$ is conditionally R-subgaussian for a fixed $R>0$. This implies that
\begin{align}
    \mathbb{E}[R_t|X_t=x_t] = \mu(x_t^T\theta_*).
\end{align}

 
\section{$c_{\mu}$ independent lower bound}
We are interested in the case where the link function is Lipschitz, but might have a very small slope. For any finite time horizon $T$ and monotonously increasing function $\mu$, we can construct an infinitesimally larger function $\mu'$, that is strictly increasing and indistinguishable from $\mu$ w.h.p. in the given horizon. Therefore for a $c_\mu$ independent lower bound, it is sufficient to show a lower bound for a $\mu$ that is only restricted by $\dot{\mu}\geq 0$.
\begin{theorem}
For any finite time horizon $T>4^5$ there exists a generalized linear bandit problem with finitely many arms, a 1-Lipschitz link function $\mu$, such that the regret of any algorithm will be at least of the order $\Omega(T^{\frac{1}{2}+\frac{1}{10}})$.
\end{theorem}
\begin{corollary}
It is impossible to derive a $c_\mu$ independent upper bound for the generalized linear bandit of the order $\tilde{\mathcal{O}}(\sqrt{T})$.
\end{corollary}
\begin{proof}
    Define the GLB Problem $P(T)$ as follows:\\
    We chose the link function 
    \begin{align}
        \mu(x) := \max\{0, x+\epsilon-1\}
    \end{align}
    with $\epsilon = T^{-\frac{1}{2}+\frac{1}{10}}$.
    We have $T^{\frac{1}{5}}$ many arms that are distributed uniformly on the 2d unit ball.\\
    $\theta_*$ is one arm chosen at random.
    \begin{lemma}For the problem $P(T)$, $T>4^5$ it holds: $\forall \, x_i\neq\theta_*:\,\mathbb{E}[R_i|X_i=x_i]=0$.
    \end{lemma}
    \begin{proof}
        For any arm $x_i\neq \theta_*$, uniform distribution of the arms implies
        \begin{align*}
            x_i^T\theta_* \leq \cos(2\pi T^{-\frac{1}{5}})
        \end{align*}
        For any $|x|\leq \frac{\pi}{2}$, simple analysis shows that $\cos(x)\leq 1-(\frac{2x}{\pi})^2$. That implies that for $T\geq 4^5$, we have
        \begin{align*}
            1-x_i^T\theta_* \geq (4T^{-\frac{1}{5}})^2 = 16 T^{-\frac{2}{5}} > \epsilon 
        \end{align*}
        Therefore $x_i^T\theta_* +\epsilon-1<0$ and $\mu(x_i^T\theta_*)=0$.
    \end{proof}
    Assume an oracle tells the agent that the set of possible values for $\theta_*$ is restricted to the set of arms. Then this problem is equivalent to a k-armed bandit problem, where each arm has expected reward 0, but one good arm has expected reward $\epsilon$. Given $\epsilon=\sqrt{\frac{k}{T}}$, the lower bound for k-armed bandits shows in this case that any algorithm will at least observe a regret of the order $\Omega(\sqrt{\frac{k}{T}})$. For any $T\geq4^5$, we can construct a problem of this kind with $k=T^{\frac{1}{5}}$. As oracle information cannot make an algorithm perform worse, so we obtain a lower bound of $\Omega(T^{\frac{1}{2}+\frac{1}{10}})$
\end{proof}
 
%% \section{$c_{\mu}$ independent lower bound for pseudo maximum-likelihood estimator based algorithms}
We are looking at an explore then exploit strategy with the pseudo maximum likelihood estimator, i.e.
\begin{align}
    \sum_{k=1}^t(\mu(x_k^T\hat{\theta})-\mu(x_k^T\theta_*)-\eta_k)x_k = 0
\end{align}
\begin{theorem}
For any finite time horizon $T$, there exists a generalized linear bandit problem such that after $T$ uniform exploration steps, the regret for playing the optimal arm for $\hat{\theta}$ will be $T^{-\frac{1}{4}}$ with a $T$ independent constant probability.
\end{theorem}
\begin{corollary}
The worst case regret for this algorithm is at least $T^{\frac{3}{4}}$.
\end{corollary}
\begin{proof}
    Define the GLB Problem as follows:\\
    We chose the link function 
    \begin{align}
        \mu(x) := \max\{0, x+\epsilon-1\}
    \end{align}
    The arms lay on the 2d unit ball.\\
    $\theta_*$ is one arm chosen at random.
    After $T$ steps of uniform exploration, we can approximate this for large T with a continuous exploration:
    \begin{align*}
        \sum_{k=1}^T\mu(x_k^T\theta_*)x_k&\approx \frac{T}{2\pi}\int_0^{2\pi} \mu(\cos(x))\cos(x)dx\theta_*\\
        &=\frac{T}{2\pi}\int_{-\cos^{-1}(1-\epsilon)}^{\cos^{-1}(1-\epsilon)} (\cos(x)-1+\epsilon)\cos(x)dx\theta_*\\
        &=\frac{T}{\pi}\left[4\epsilon^{\frac{3}{2}}\sqrt{2-\epsilon}+(2\cos^{-1}(1-\epsilon)+\sin(2\cos^{-1}(1-\epsilon))-4\sin(\cos^{-1}(1-\epsilon))\right]\theta_*\\
        &=\frac{T}{\pi}\left[4\epsilon^{\frac{3}{2}}\sqrt{2-\epsilon} -\mathcal{O}(\cos^{-1}(1-\epsilon)^3)\right]\theta_*
    \end{align*}
    Up to constants and high order terms of the order $\mathcal{O}(\epsilon^\frac{5}{2})$, we have
    \begin{align*}
        \sum_{k=1}^T\mu(x_k^T\theta_*)x_k&\approx T\epsilon^{\frac{3}{2}}\theta_*
    \end{align*}
    Due to symmetry, it also holds that $\sum_{k=1}^T\mu(x_k^T\hat{\theta})x_k$ is collinear to $\hat{\theta}$.\\
    Let $\bar{\theta}$ be orthogonal to $\hat{\theta}$ and of length 1, then
    \begin{align*}
        (\sum_{k=1}^t(\mu(x_k^T\hat{\theta})-\mu(x_k^T\theta_*)-\eta_k)x_k)^T\bar{\theta}=0\\
        T\epsilon^{\frac{3}{2}}\bar{\theta}^T\theta_* \approx \sum_{k=1}^T\eta_k x_k^T\bar{\theta}
    \end{align*}
    We assume the noise is gaussian with variance 1, then the RHS is a gaussian random variable with variance approximately
    \begin{align*}
        \mathbb{V}\left[\sum_{k=1}^T\eta_k x_k^T\bar{\theta}\right] \approx \frac{T}{\pi}\int_0^\pi \cos^2(x)dx =\frac{T}{2}
    \end{align*}
    With a constant probability, it therefore holds up to constants
    \begin{align*}
        \bar{\theta}^T\theta_* \geq \sqrt{\epsilon^{-3}T^{-1}}\\
    \end{align*}
    Setting $\epsilon= T^{\frac{-1}{4}}$, we have up to constants
    \begin{align*}
        \bar{\theta}^T\theta_* \geq \sqrt{\epsilon} \approx \sqrt{\epsilon(2-\epsilon)}\\
    \hat{x}^T\theta_* = \sqrt{1-(\bar{\theta}^T\theta_*)^2} \leq 1-\epsilon
    \end{align*}
    The immediate regret after exploration is therefore at least $\epsilon =T^{-\frac{1}{4}}$
    
\end{proof}
 %% if we want to show that there is a gap between Least Square and pseudo ML

\section{$\mu$ independent upper bound for Explore then Commit with least square estimator}
\subsection{Problem assumptions}
The following assumptions are currently made, 
\begin{itemize}
\item the arms consist of the surface of the unit ball $\mathcal{B}_1$ (scaling can be absorbed in $\kappa$).
\item we are playing the arms for a total of $T$ time-steps and this value is known beforehand.
\item the norm of $||\theta_*||$ is fixed to 1. (scaling can be absorbed in $\kappa$). We can further use this value in our exploration budget. \color{red}should be generalized to $||\theta_*||\leq 1$ later.\color{black}
\item $0\leq\dot{\mu}\leq \kappa$ \color{red}RHS will be eventually replaced by $\mu(x)\geq \mu(||\theta_*||)-\kappa(1-x)||\theta_*||$\color{black}
\item $\dot{\mu}(x)=\dot{\mu}(-x)$ \color{red}shouldn't be required, but also doesn't hurt much and makes the analysis much easier\color{black}
\item the noise is $\sigma$-subgaussian 
\end{itemize}

We are further defining the following functions
\begin{align}
    L(\theta) &= \EV{(\mu(x^T\theta)-R)^2}\nonumber\\
    &= \EV{(\mu(x^T\theta)-\mu(x^T\theta_*)-\eta)^2}=\EV{(\mu(x^T\theta)-\mu(x^T\theta_*))^2}+\sigma^2\\
    L_n(\theta) &= \frac{1}{n}\sum_{k=1}^n(\mu(x_k^T\theta)-R_k)^2= \frac{1}{n}\sum_{k=1}^n(\mu(x_k^T\theta)-\mu(x_k^T\theta_*)-\eta_k)^2
\end{align}
Obviously $\theta_*$ is the minimum of $L$.\\
The least square estimator is
\begin{align}
    \hat{\theta} := \arg\,\min_{\theta} L_n(\theta)
\end{align}
\textbf{Our algorithm is Explore-then-Commit:}\\
Explore the arms uniformly until a stopping time $\mathcal{T}$ \color{red}currently this is a fixed value because we know the length of $||\theta_*||$ \color{black}. After this commit to playing $\hat{\theta}$ for the remaining time-steps.
\subsection{Results}
\color{red}This is what we aim for, the proof still has gaps though\color{black}
\begin{theorem}
    For any time $T$ and any $\mu$ under the given constraints, with probability at least $1-\delta$, the regret of the given algorithm will be bounded by
    \begin{align}
        \operatorname{Reg}(T) \leq c T\caret\left(\frac{1}{2}+\frac{d-1}{2d+6}\right)\cdot(\kappa\sigma^2\left(\log(T)^2+\log(\delta^{-1})\right))\color{red}^p\color{black}
    \end{align}
\end{theorem}
The proof will require the following lemmas:
\begin{lemma}
    With probability at least $1-\delta$, is holds that 
    \begin{align}
    L(\hat{\theta})-L(\theta_*)\leq \frac{8}{n}\sigma^2(c_1\log(2n)+c_2\log{\delta^{-1}})+\color{red}?\color{black}
    \end{align}
\end{lemma}
\begin{proof}
    As $\hat{\theta}$ is the minimizer of $L_n$, it holds that
    \begin{align*}
        \frac{1}{n}\sum_{k=1}^n(\mu(x_k^T\hat{\theta})-\mu(x_k^T\theta_*))^2 \leq \frac{1}{n}\sum_{k=1}^n(\mu(x_k^T\hat{\theta})-\mu(x_k^T\theta_*))^2 + L_n(\theta_*)-L_n(\hat{\theta})\\
        = \frac{2}{n}\sum_{k=1}^n(\mu(x_k^T\hat{\theta})-\mu(x_k^T\theta_*))\eta_k\\
        = \frac{2}{\sqrt{n}}\sqrt{\frac{1}{n}\sum_{k=1}^n(\mu(x_k^T\hat{\theta})-\mu(x_k^T\theta_*))^2}\sum_{k=1}^n\frac{(\mu(x_k^T\hat{\theta})-\mu(x_k^T\theta_*))}{\sqrt{\sum_{l=1}^n(\mu(x_l^T\hat{\theta})-\mu(x_l^T\theta_*))^2}}\eta_k\\
        w.h.p. \leq   \frac{2}{\sqrt{n}}\sqrt{ \frac{1}{n}\sum_{k=1}^n(\mu(x_k^T\hat{\theta})-\mu(x_k^T\theta_*))^2 }\sigma(c_1\sqrt{\log(2n)}+c_2\sqrt{\log{\delta^{-1}}})\\
        \frac{1}{n}\sum_{k=1}^n(\mu(x_k^T\hat{\theta})-\mu(x_k^T\theta_*))^2 \leq \frac{8}{n}\sigma^2(c_1\log(2n)+c_2\log{\delta^{-1}})
    \end{align*}
    \color{red}TODO: bound $\EV{(\mu(x^T\theta)-\mu(x^T\theta_*))^2}-\frac{1}{n}\sum_{k=1}^n(\mu(x_k^T\hat{\theta})-\mu(x_k^T\theta_*))^2$ for bound on $L(\hat{\theta})-L(\theta_*)$. Write down the constants $c_1, c_2$ explicitly.\color{black}\\
\end{proof}
\begin{lemma}
    Let $\tilde{\theta}$ be the vector $\hat{\theta}$ rescaled such that it matches the length of $\theta_*$. Then 
    $L(\hat{\theta})-L(\theta_*) \geq \frac{1}{2}(L(\tilde{\theta})-L(\theta_*))$
\end{lemma}
\begin{proof}
Let $\theta_1$ and $\theta_2$ be the two orthogonal vectors such that $\theta_* = \theta_1-\theta_2$ and $\tilde{\theta}=\theta_1+\theta_2$.\\
Define the two sets
\begin{align*}
\mathcal{B}_1 &= \left\{x|\sign(x^T\theta_1)=\sign(x^T\theta_2)\right\}\\ 
\mathcal{B}_2 &= \left\{x|\sign(x^T\theta_1)=-\sign(x^T\theta_2)\right\}.
\end{align*}
Due to Symmetry it holds that 
$$\EV{(\mu(x^T\tilde{\theta})-\mu(x^T\theta_*))^2|x\in\mathcal{B}_1}=\EV{(\mu(x^T\tilde{\theta})-\mu(x^T\theta_*))^2|x\in\mathcal{B}_2}$$
If $||\hat{\theta}||_2 > ||\tilde{\theta}||_2$, then $(\mu(x^T\hat{\theta})-\mu(x^T\theta_*))^2 \geq (\mu(x^T\tilde{\theta})-\mu(x^T\theta_*))^2$ for all $x\in \mathcal{B}_1$.\\
If $||\hat{\theta}||_2 < ||\tilde{\theta}||_2$, then $(\mu(x^T\hat{\theta})-\mu(x^T\theta_*))^2 \geq (\mu(x^T\tilde{\theta})-\mu(x^T\theta_*))^2$ for all $x\in \mathcal{B}_2$.\\
Finally we get
\begin{align*}
     &L(\hat{\theta})-L(\theta_*) = \EV{(\mu(x^T\hat{\theta})-\mu(x^T\theta_*))^2}\\
     =& \frac{1}{2}\left(\EV{(\mu(x^T\hat{\theta})-\mu(x^T\theta_*))^2|x\in\mathcal{B}_1}+\EV{(\mu(x^T\hat{\theta})-\mu(x^T\theta_*))^2}|x\in\mathcal{B}_2\right)\\
     \geq& \frac{1}{2}\EV{(\mu(x^T\tilde{\theta})-\mu(x^T\theta_*))^2}=\frac{1}{2}(L(\tilde{\theta})-L(\theta_*))
\end{align*}
\end{proof}
\subsubsection{$d=2$}
\color{red}This is the most important gap. It holds for $\mu = (x-1+\epsilon)_+$, but I haven't succeeded in a general proof.\color{black}
\begin{proposition}
Let $d=2$. Then for any valid $\mu \in M$, it holds that 
\begin{align*}
    L(\tilde{\theta})-L(\theta_*) \geq \kappa^{-\frac{1}{2}}(\mu(x_*^T\theta_*)-\mu(\hat{x}^T\theta_*))(\mu(1)-\mu(-1))^{\frac{3}{2}}
\end{align*}
\end{proposition}
The proof follows from the following Lemma
\begin{lemma}
    For any $\mu\in M_1$ and $z \in [0,1]$, it holds that
\begin{align*}
\int_{-1}^1&\left(\mu\left(zx + \iz\ix\right)-\mu\left(zx - \iz\ix\right)\right)^2\frac{1}{\ix}\,dx \\
&\geq c \left(\mu(1)-\mu(z^2)\right)\left(\mu(1)-\mu(-1)\right)^\frac{3}{2}\\
& \Leftrightarrow\\
\int_{-1}^1&(1-z^2)\left(\int_{-1}^1\dot{\mu}\left(zx+\iz\ix y\right)\,dy\right)^2\ix\,dx \\
&\geq c \left(\int_{z^2}^1\dot{\mu}(x)\,dx\right)\left(\int_{-1}^1\dot{\mu}(x)\,dx\right)^\frac{3}{2}
\end{align*}
\end{lemma}
\begin{proof}[Proof of Theorem 6]

\begin{align*}
    L(\tilde{\theta})-L(\theta_*)
    =&\EV{(\mu(x^T\theta_1-x^T\theta_2)-\mu(x^T\theta_1+x^T\theta_2))^2}\\
    =&\int_{0}^{2\pi}\left(\mu(\cos(x)|\theta_1|-\sin(x)|\theta_2|)-\mu(\cos(x)|\theta_1|+\sin(x)|\theta_2|)\right)^2\, dx\\
    =&2\int_{0}^{\pi}\left(\int_{-\sin(x)|\theta_2|}^{\sin(x)|\theta_2|}\dot{\mu}(\cos(x)|\theta_1|+y)\,dy\right)^2\, dx\\
    =&2|\theta_2|^2\int_{0}^{\pi}\left(\int_{-1}^{1}\dot{\mu}(\cos(x)|\theta_1|+\sin(x)|\theta_2|y)\,dy\right)^2\sin(x)^2\, dx\\
    =&2|\theta_2|^2\int_{-1}^{1}\left(\int_{-1}^{1}\dot{\mu}(x|\theta_1|+\sqrt{1-x^2}|\theta_2|y)\,dy\right)^2\sqrt{1-x^2}\, dx\\
...
\end{align*}
\end{proof}

\subsection{$d\geq 3$}
We express the Expectation in terms of integrals
\begin{align*}
   & L(\tilde{\theta})-L(\theta_*)\\
    =&\EV{(\mu(x^T\theta_1-x^T\theta_2)-\mu(x^T\theta_1+x^T\theta_2))^2}\\
    =&\frac{1}{|S_d|}\int_{S_d}(\mu(x^T\theta_1-x^T\theta_2)-\mu(x^T\theta_1+x^T\theta_2))^2 \,d\mathcal{L}_{d-1}(x)\\
    =& \frac{2|S_{d-2}|}{|S_d|}\int_{0}^{\frac{\pi}{2}}\cos(\alpha)^{d-2}\\
    &\,\int_{0}^{2\pi}(\mu(\sin(\alpha)(\cos(x)|\theta_1|-\sin(x)|\theta_2|))-\mu(\sin(\alpha)(\cos(x)|\theta_1|+\sin(x)|\theta_2|)))^2 \,dx\,d\alpha\\
    \geq& \frac{4|S_{d-2}|}{|S_d|}\int_{0}^{\frac{\pi}{2}}\cos(\alpha)^{d-2}
    J(\mu,\sin(\alpha)||\theta_*||)\,d\alpha\\
    =& \frac{4|S_{d-2}||\theta_2|^2}{|S_d|}\int_{0}^{1}\frac{\alpha^{d-2}}{\sqrt{1-\alpha^2}}
    J(\mu,\sqrt{1-\alpha^2})||\theta_*||)\,d\alpha\\
\end{align*}


\end{document}
