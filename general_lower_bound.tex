\section{$c_{\mu}$ independent lower bound}
We are interested in the case where the link function is Lipschitz, but might have a very small slope. For any finite time horizon $T$ and monotonously increasing function $\mu$, we can construct an infinitesimally larger function $\mu'$, that is strictly increasing and indistinguishable from $\mu$ w.h.p. in the given horizon. Therefore for a $c_\mu$ independent lower bound, it is sufficient to show a lower bound for a $\mu$ that is only restricted by $\dot{\mu}\geq 0$.
\begin{theorem}
For any finite time horizon $T>4^5$ there exists a generalized linear bandit problem with finitely many arms, a 1-Lipschitz link function $\mu$, such that the regret of any algorithm will be at least of the order $\Omega(T^{\frac{1}{2}+\frac{1}{10}})$.
\end{theorem}
\begin{corollary}
It is impossible to derive a $c_\mu$ independent upper bound for the generalized linear bandit of the order $\tilde{\mathcal{O}}(\sqrt{T})$.
\end{corollary}
\begin{proof}
    Define the GLB Problem $P(T)$ as follows:\\
    We chose the link function 
    \begin{align}
        \mu(x) := \max\{0, x+\epsilon-1\}
    \end{align}
    with $\epsilon = T^{-\frac{1}{2}+\frac{1}{10}}$.
    We have $T^{\frac{1}{5}}$ many arms that are distributed uniformly on the 2d unit ball.\\
    $\theta_*$ is one arm chosen at random.
    \begin{lemma}For the problem $P(T)$, $T>4^5$ it holds: $\forall \, x_i\neq\theta_*:\,\mathbb{E}[R_i|X_i=x_i]=0$.
    \end{lemma}
    \begin{proof}
        For any arm $x_i\neq \theta_*$, uniform distribution of the arms implies
        \begin{align*}
            x_i^T\theta_* \leq \cos(2\pi T^{-\frac{1}{5}})
        \end{align*}
        For any $|x|\leq \frac{\pi}{2}$, simple analysis shows that $\cos(x)\leq 1-(\frac{2x}{\pi})^2$. That implies that for $T\geq 4^5$, we have
        \begin{align*}
            1-x_i^T\theta_* \geq (4T^{-\frac{1}{5}})^2 = 16 T^{-\frac{2}{5}} > \epsilon 
        \end{align*}
        Therefore $x_i^T\theta_* +\epsilon-1<0$ and $\mu(x_i^T\theta_*)=0$.
    \end{proof}
    Assume an oracle tells the agent that the set of possible values for $\theta_*$ is restricted to the set of arms. Then this problem is equivalent to a k-armed bandit problem, where each arm has expected reward 0, but one good arm has expected reward $\epsilon$. Given $\epsilon=\sqrt{\frac{k}{T}}$, the lower bound for k-armed bandits shows in this case that any algorithm will at least observe a regret of the order $\Omega(\sqrt{\frac{k}{T}})$. For any $T\geq4^5$, we can construct a problem of this kind with $k=T^{\frac{1}{5}}$. As oracle information cannot make an algorithm perform worse, so we obtain a lower bound of $\Omega(T^{\frac{1}{2}+\frac{1}{10}})$
\end{proof}